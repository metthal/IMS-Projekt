%%%%%%%%%%%%%%%%%%%%%%%%%%%%%%%%%%%%%%%%%%%%%%%%%%%%%%%%%%%%%%%%%%%%%%%%%%%%%%
%%
%% Dokumentacia k projektu 'Interpret pre jazyk IFJ 2013'
%%
%%
%%%%%%%%%%%%%%%%%%%%%%%%%%%%%%%%%%%%%%%%%%%%%%%%%%%%%%%%%%%%%%%%%%%%%%%%%%%%%%
\documentclass[12pt,a4paper,titlepage,final]{article}

% jazyk
\usepackage[slovak]{babel}
\usepackage[utf8]{inputenc}
% balicky prr odkazy
\usepackage[bookmarksopen,colorlinks,plainpages=false,urlcolor=blue,unicode]{hyperref}
\usepackage{url}
% obrazky
\usepackage[dvipdf]{graphicx}
% velikost stranky
\usepackage[top=3.5cm, left=2.5cm, text={17cm, 24cm}, ignorefoot]{geometry}
\usepackage{subfigure}

\begin{document}

%%%%%%%%%%%%%%%%%%%%%%%%%%%%%%%%%%%%%%%%%%%%%%%%%%%%%%%%%%%%%%%%%%%%%%%%%%%%%%
% titulní strana

\def\projname{Okruh 4. Výrobný podnik}


\begin{titlepage}

% \vspace*{1cm}
\begin{figure}[!h]
  \centering
  \includegraphics[height=5cm]{logo.eps}
\end{figure}
\center Fakulta Informačních Technologií \\
\center Vysoké Učení Technické v Brně \\

\vfill

\begin{center}
\bigskip
\begin{Huge}
\projname\\
\end{Huge}
\begin{large}
Modelovanie a simulácie\\
\end{large}
\end{center}

\vfill

\begin{center}
\begin{Large}
\today
\end{Large}
\end{center}

\vfill

\begin{flushleft}
\begin{large}
Marek Milkovič (xmilko01)\\
Lukáš Vrabec (xvrabe07) \\
\end{large}
\end{flushleft}
\end{titlepage}

%%%%%%%%%%%%%%%%%%%%%%%%%%%%%%%%%%%%%%%%%%%%%%%%%%%%%%%%%%%%%%%%%%%%%%%%%%%%%%
% obsah
\pagestyle{plain}
\pagenumbering{gobble}
\tableofcontents

%%%%%%%%%%%%%%%%%%%%%%%%%%%%%%%%%%%%%%%%%%%%%%%%%%%%%%%%%%%%%%%%%%%%%%%%%%%%%%
% textova zprava
\newpage
\pagestyle{plain}
\pagenumbering{arabic}
\setcounter{page}{1}

% sekcia 1
%%%%%%%%%%%%%%%%%%%%%%%%%%%%%%%%%%%%%%%%%%%%%%%%%%%%%%%%%%%%%%%%%%%%%%%%%%%%%%
\section{Úvod}
V tejto práci je riešená implementácia modelu hypotetického výrobného podniku
na produkciu základných dosiek, ktorá bude použitá pre vytvorenie simulačného
modelu systému.

Na základe modelu a simulačných experimentov bude ukázané chovanie systému
výrobného podniku v podmienkach rôzneho vyťaženia, pridávania jednotlivých
obslužních liniek.
TODO: scim experimentujeme. 

Cieľom experimentov je demonštrovanie možných optimalizácií na zadanom modely. Kon\-kré\-tne
Konkrétne nájdenie optimálnejšej konfigurácie vzhľadom na počet vyrobených
kusov základných dosiek ročne pri čo najmenšej veľkosti a cene výslednej 
výrobnej linky.

Pre spracovanie modelu je nutné naštudovať a pochopiť samotný proces výroby
základnej dosky(ďalej len výroby). Vyhľadať a spracovať informácie týkajúce sa 
technologií ktoré sú použité pri výrobe. Získať technické špecifikácie zariadení,
ktoré sa podieľajú na výrobe. Pri\-spô\-so\-be\-nie modelovaného systému tak,
Prispôsobenie modelovaného systému tak, aby čo najviac odpovedal reálnemu
systému výroby. Následne spracovať tieto údaje, navrhnúť experimenty pré 
vytvorenie optimálnejších riešení výroby. 

%%%%%%%%%%%%%%%%%%%%%%%%%%%%%%%%%%%%%%%%%%%%%%%%%%%%%%%%%%%%%%%%%%%%%%%%%%%%%%
\subsection{Autori a zdroje faktov}
Na projekte sa podieľali študenti VUT FIT Marek Milkovič a Lukáš Vrabec. Informácie
a odborné fakty boli čerpané z internetu, odborných článkov a výročných
správ spoločností GIGA-BYTE Technology Co. Ltd. a ASUSTeK Computer Inc.
Všetky zdroje sú uverejnené na konci tohto dokumentu.

\subsection{Overenie validity modelu}
Validita modelu bola overovaná porovnávaním dát z dostupných zdrojov s výstupmi
simulácie. Pri vytváraní modelu boli čiastočne použité dôverihodné informácie,
pre nedôverihodné alebo nedostupné informácie boli zvolené hypotetické hodnoty,
ktoré približne odpovedajú reálnemu systému. 

% sekcia 2
%%%%%%%%%%%%%%%%%%%%%%%%%%%%%%%%%%%%%%%%%%%%%%%%%%%%%%%%%%%%%%%%%%%%%%%%%%%%%%
\section{Rozbor témy a použitých metód}
%%%%%%%%%%%%%%%%%%%%%%%%%%%%%%%%%%%%%%%%%%%%%%%%%%%%%%%%%%%%%%%%%%%%%%%%%%%%%%
Výrobný podnik spracováva požiadavky na výrobu základných dosiek, ktorý pracuje
v nepretržitej prevádzke. Prvým krokom v procese výroby je vstup dosiek plošných
spojov (\textit{angl.} printable circuit board, ďalej len PCB) do SMT
(\textit{angl. surface mount-technology}) liniek, kde dochádza k osadeniu jednotlivých
SMD (\textit{angl. surface-mount device}) komponent (rezistory, chipset atď.).
SMT linky sa skladajú zo zariadenia pre sieťotlač spájkovacej 
hmoty (\textit{angl. solder paste screen printer}) kde dochádza k nanášaniu spájkovacej
hmoty na PCB. Tento proces je vykonávaný 2 až 150 mm/sec [link]. Ďalej pokračuje
do P\&P (\textit{angl. Pick-and-Place}) prístroja ktorý na PCB umiestni 
SMD komponenty. Rýchlosť umiestnia SMD komponent na PCB je 0.25 až 0.5 sekundy
na čip [link]. Osadené komponenty je nutné upevniť. K tomuto účelu sa použije
pec ktorá pri vysokých teplotách upevní komponenty (spájkovanie pretavením), 
trvanie približne 5 minút v závisloti na použitom materiále. 

Proces pokračuje kontrolou nanometrových chýb spojov na PCB. Tento proces je 
nazývaný AOI (\textit{angl. automated optical insepction}). V prípade výskytu
chýb je PCB nutné manuálne opraviť. Po oprave je nutné AOI zopakovať. Tento
úkon nieje časovo náročny [link].

Následne je PCB umiestnená na DIP (\textit{angl. Dual in-line package})
linku ktorú predstavuje posuvný pás na ktorom sú manuálne umiesňované 
DIP konektory (DIMM,PCI-Express atď.). DIP konetkory sú upevnené v špeciálnej
peci (spájkovanie vlnou).

Proces výroby základnej dosky je hotový. Vykoná sa viacero manuálnych aj automatizovaných
testov. V prípade chýb je základná doska manuálne opravovaná a sú na nej opakovane vykonané
testy.

Posledným krokom je zabalenie základnej dosky, ktorá je pripravená na distribúciu.

Výrobný podnik môže obsahovať niekoľko SMT, DIP, testovích a baliacích liniek.

\subsection{Popis použitých postupov}
Simulačný model je konzolová aplikácia, naprogramovaná v programovacom jazyku
C++, podľa požiadaviek zadania. Simulačná knižnica SIMLIB/C++ bola použitá pretože
umožnuje objektovo zapísať model, čím je implementácia jednoduchšia a efektívnejšia.
Z tejto knižnice boli použité aj simulačné nástroje na zber štatistík. Na tvorbu
tabuliek a grafov bol použitý tabuľkový procesor Microsoft Excel.

\subsection{Popis pôvodu použitých metód}
Bola použitá knižnica SIMLIB/C++, v ktorej sme rozšírili model zariádení a 
skladov o identifikáciu v ktorej výrobnej linke sa nachádza.

% sekcia 3
%%%%%%%%%%%%%%%%%%%%%%%%%%%%%%%%%%%%%%%%%%%%%%%%%%%%%%%%%%%%%%%%%%%%%%%%%%%%%%
\section{Koncepcia modelu}
%%%%%%%%%%%%%%%%%%%%%%%%%%%%%%%%%%%%%%%%%%%%%%%%%%%%%%%%%%%%%%%%%%%%%%%%%%%%%%

Podľa dostupných informáčních zdrojov sa predpokladá ročná výroba 7 miliónov
základných dosiek na továreň, čo je približne 21 miliónov základných dosiek
na spoločnosť. Z toho dôvodu je nutné vytvoriť požiadavku na 19179 kusov deňne.
PCB sa radia do spoločnej fronty pred vstupom do SMT liniek. Zvolená je tá SMT
linka na ktorej sa náchádza volné zariadenie pre sieťotlač spájkovacej hmoty.
Predpokladáme rovnomerné trvanie sieťotlače 30 až 45 sekúnd pri nastavení sieťotlače
na 10mm/sec a rozmeroch PCB 305x244 mm (ATX model).

PCB sa ďalej radia do fronty P\&P zariadenia v SMT linke v ktorej sa nachádzajú.
Predpoklad je, že P\&P zariadenie osadí čipy rýchlosťou 0.25 sekundy na čip, pričom
sa predpokladajú hypotetické základné dosky obsahujúce 240 až 480 SMD komponent
z dôvodu nedostatku informácií. Tento proces bude trvať rovnomerne 1 až 2 minuty.
PCB sa pohybujú po výrobnej linke za sebou cez pec, tento prechod trvá 5 minút.

Ďalej je zahájený proces AOI zariadení s frontov ktorého trvanie je rovnomerne 10 až 14 sekúnd, v 
prípade že nastane chyba s pravedepodobnosťou 1\% je doska opravovaná. Dĺžka
opravy je daná exponencionálnym rozložením so stredom 6 minuty. Opravené PCB
ktoré musia znova podstúpiť AOI majú vyššiu prioritu ako PCB vstupujúce do AOI 
po prvý krát.

Pri procese prechodu z SMT linky na DIP linku, je vybraná prvá DIP linka s dostupnou
kapacitou, v prípade plného obsadenia DIP liniek je vybraná tá s najkratšou
frontou. Neskôr dochádza k osadzovaniu DIP konetkorov na výrobnej linke
ktorá predstavuje posuvný pás a následne musí základná doska prejsť ďalšou
pecou. Nakoľko je tento proces vykonávaný na jednom posuvnom páse spojenie týchto
procesov nijak neovplivní výsledky modelu. Celý tento proces trvá 2 minuty.

Pri prechode z DIP liniek na testovacie linky je použitý rovnáky mechanizmus
výberu linky ako pri prechode z SMT liniek na DIP linky. V testovacej fáze 
dochádza k manuálnym a automatizovaným testom. Dĺžka testovania je daná 
exponencionálnym rozložením so stredom 2 minuty, pričom v 1\% prípadov dochádza
k potrebe dosku opraviť, dĺžka opravy je daná exponencionálnym rozložením so
stredom 3 minuty. Opravené zákldné dosky musia podstúpiť testovanie znova, pričom
majú vyššiu prioritu.

Nasleduje prechod základných dosiek na baliace linky. Mechanizmus prechodu je
rovnaký ako v prípade prechodu z DIP liniek na testovacie linky. Balenie trvá 
rovnomerne 40 až 60 sekúnd.

V modely sa nepredpokladá s možnou poruchou na zariadeniach z dôvodu nízkej
poruchovosti.

\subsection{Návrh konceptuálneho modelu}
Požiadavok na výrobu základnej dosky vstupuje do systému kde sa zaraďuje do fronty
SMT liniek. Postupuje cez solder paste screen printer, P\&P a AOI, ktoré 
sú modelované ako zariadenia. Presúva sa na DIP linku, následne na testovaciu
a baliacu linku ktoré sú modelované ako sklad. Následne proces predstavujúci
hotovú základnú dosku opúšta systém.

\subsection{Formy konceptuálneho modelu}
Linky sú znazornené pomocou samotných petriho sietí, nakoľko prechody medzi linkami
nebolo možné namodelovať a jednotlivých liniek može byť v systéme niekoľko.
Jednotlivé petriho siete sa napájajú na prázdne miesto na konci každej
siete. 

\begin{figure}[!h]
  \centering
  \includegraphics[width=18cm]{smt.png}
  \caption{SMT linka}
\end{figure}

\begin{figure}[!h]
  \centering
  \includegraphics[width=10cm]{dip.png}
  \caption{DIP linka}
\end{figure}

\begin{figure}[!h]
  \centering
  \includegraphics[width=14cm]{tst.png}
  \caption{Testovacia linka}
\end{figure}

\begin{figure}[!h]
  \centering
  \includegraphics[width=10cm]{pkg.png}
  \caption{Baliaca linka}
\end{figure}

% sekcia 4
%%%%%%%%%%%%%%%%%%%%%%%%%%%%%%%%%%%%%%%%%%%%%%%%%%%%%%%%%%%%%%%%%%%%%%%%%%%%%%
\section{Architektúra simulačného modelu}
%%%%%%%%%%%%%%%%%%%%%%%%%%%%%%%%%%%%%%%%%%%%%%%%%%%%%%%%%%%%%%%%%%%%%%%%%%%%%%
V implementácií boli použité triedy knižnice SIMLIB/C++. Ako generátor požiadavkov
na výrobu bola použitá trieda \texttt{Event}, tento generátor je aktivovaný počas simulácie
v čase 0, následne je spušťaný každých 24 hodín. Generátor vytvára požiadavky typu
\texttt{Board}, ktoré dedia triedu \texttt{Process}.

Doba simulácie je nastavená na 1 deň. Časová jednotka použitá v simulácií je 
sekunda.

\subsection{Mapovanie abstraktného modelu do simulačného modelu}
Všetky zariadenia na výrobných linkách sú odvodené od triedy \texttt{Machine}
a všetky sklady od triedy \texttt{Line}. Obe tieto triedy umožnujú identifikovať
na ktorých výrobných linkách sa nachádzajú. Generátor požiadavkov na výrobu základných
dosiek je implementovaný v triede \texttt{Generator}. Trieda \texttt{Board} predstavuje
základnú dosku od počiatku výrobného procesu až po jej zabalenie. 

\subsection{Použitie programu}
Simulácia sa kompiluje pomocou príkazu \texttt{\$ make}. Sputenie experimentov
je prevádzané pomocou príkazu \texttt{\$ make run}. Parametre umožnujúce modifikovať
simulačný model sú nasledovné.

\begin{table}[h]
\centering
\begin{tabular}{|l|l|}
\hline
\multicolumn{1}{|c|}{\textbf{Parameter}} & \multicolumn{1}{c|}{\textbf{Popis}}  \\ \hline
\texttt{--smt N}    & Pri sumlácií bude použitých N SMT liniek.                          \\ \hline
\texttt{--dip N}    & Pri sumlácií bude použitých N DIP liniek.                          \\ \hline
\texttt{--tst N}    & Pri sumlácií bude použitých N testovacích liniek.                  \\ \hline
\texttt{--pkg N}    & Pri sumlácií bude použitých N baliacich liniek.                    \\ \hline
\texttt{--req N}    & Bude vygenerovaných N požiadavkov za deň.                          \\ \hline
\texttt{--pnp N M}  & Trvanie výroby na P\&P stroji bude rovnomerne N až M sekúnd.       \\ \hline
\texttt{--out FILE} & Výstup bude presmerovaný do súboru FILE.                           \\ \hline
\texttt{--err N M}  & Chyba na AOI nastane v N\% prípadov, pri testovaní v M\% prípadov. \\ \hline
\end{tabular}
\end{table}


% sekcia 5
%%%%%%%%%%%%%%%%%%%%%%%%%%%%%%%%%%%%%%%%%%%%%%%%%%%%%%%%%%%%%%%%%%%%%%%%%%%%%%
\section{Simulačné experimenty a ich priebeh}
%%%%%%%%%%%%%%%%%%%%%%%%%%%%%%%%%%%%%%%%%%%%%%%%%%%%%%%%%%%%%%%%%%%%%%%%%%%%%%

% sekcia 6
%%%%%%%%%%%%%%%%%%%%%%%%%%%%%%%%%%%%%%%%%%%%%%%%%%%%%%%%%%%%%%%%%%%%%%%%%%%%%%
\section{Zhrnutie simulačných experimentov a záver}
%%%%%%%%%%%%%%%%%%%%%%%%%%%%%%%%%%%%%%%%%%%%%%%%%%%%%%%%%%%%%%%%%%%%%%%%%%%%%%


\end{document}
